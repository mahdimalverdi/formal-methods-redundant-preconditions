% !TeX program = xelatex
\documentclass[11pt,a4paper]{article}

\usepackage[margin=2.5cm]{geometry}
\usepackage{hyperref}
\usepackage{xcolor}
\usepackage{graphicx}
\usepackage{booktabs}
\usepackage{enumitem}
\usepackage{listings}
\usepackage{fancyhdr}

% Persian / RTL
\usepackage{xepersian}

% Vazir-family (Vazirmatn) — expects fonts in ../fonts/ when compiling from docs/
\settextfont[
  Path=../fonts/,
  BoldFont=Vazirmatn-Bold.ttf
]{Vazirmatn-Regular.ttf}
\setdigitfont[
  Path=../fonts/,
  BoldFont=Vazirmatn-Bold.ttf
]{Vazirmatn-Regular.ttf}

% Latin font (keep a safe default that exists on most systems/tectonic)
\setlatintextfont{DejaVu Sans}

\hypersetup{
  colorlinks=true,
  linkcolor=blue,
  urlcolor=blue
}

\lstset{
  basicstyle=\ttfamily\small,
  breaklines=true,
  columns=fullflexible,
  frame=single,
  rulecolor=\color{black!20}
}

\setlist[itemize]{noitemsep, topsep=0.25em}
\setlist[enumerate]{noitemsep, topsep=0.25em}

\pagestyle{fancy}
\fancyhf{}
\rhead{عنوان مقاله}
\lhead{نام نویسنده}
\cfoot{\thepage}

\title{عنوان مقاله}
\author{نام نویسنده \\ \lr{\texttt{شماره}}}
\date{\lr{2026-01-30}}

\begin{document}
\maketitle

\begin{RTL}

\begin{abstract}
خلاصه‌ی کوتاه مقاله/گزارش (حدود ۵ تا ۸ خط). در صورت نیاز از \lr{English terms} با \lr{\texttt{\textbackslash lr}} استفاده کنید.
\end{abstract}

\textbf{کلمات کلیدی:} روش‌های صوری، قراردادها، پیش‌شرط، افزونگی

\section{مقدمه}
متن مقدمه.

\section{روش/پیاده‌سازی}
توضیح روش و جزئیات اجرا.

\section{نتایج}
نتایج آزمایش/شبیه‌سازی.

\section{جمع‌بندی}
جمع‌بندی و کارهای آینده.

\section*{منابع}
\begin{enumerate}
  \item \lr{N. Thoben, H. Wehrheim, ``Detecting Redundant Preconditions,'' FormaliSE 2025.}
\end{enumerate}

\end{RTL}
\end{document}

